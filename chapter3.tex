\chapter{基本的な文章構成の原則}
\section{パラグラフを文章構成の単位にする: 各トピックにつき1パラグラフ}
もし書いている対象がわずかの広がりしか持たないか、または対象をごく手短にしか扱う気がないのであれば、複数のトピックに分割する必要はないだろう。よって、短い記述、文学作品の短い要約、ひとつの出来事についての短い説明、ひとつの行為の概略を示すだけの談話、単一の考えについての論述、これらはいずれも単一のパラグラフで書くのが最善だ。パラグラフを書いた後で、さらに分割することでより良くできるかどうかを考えればよい。
\par
しかしながら普通は、主題は複数のトピックに分ける必要がある。各トピックはパラグラフの主題となるべきものだ。各トピックをパラグラフで単独に扱う目的は、もちろん読み手を助けるためだ。各パラグラフの冒頭は読み手にとっての信号で、主題が発展する新しいステップに達したことを示す。
\par
分割の度合は、文章の長さによって異なるだろう。例えば、本や詩の短い説明は単一のパラグラフから成っているかもしれない。もう少し長いものは2つのパラグラフから成っているかもしれない:
\begin{enumerate}[label=\Alph*.]
    \item Account of the work. (作品の説明)
    \item Critical discussion. (批判的な議論)  
\end{enumerate}

文学のクラスのために書かれた詩のリポートは、7つのパラグラフから構成されうる:
\begin{enumerate}[label=\Alph*.]
    \item Facts of composition and publication. (文章と出版に関する事実関係)
    \item Kind of poem; metrical form.(詩の種類;韻文の形式)
    \item Subject.(主題)
    \item Treatment of subject.(主題の取り扱い)
    \item For what chiefly remarkable.(主にどこが注目すべきか)
    \item Wherein characteristic of the writer.(書き手の特徴が表れているのはどこか)
    \item Relationship to other works.(他の作品との関連)
\end{enumerate}
パラグラフCとDは、詩によって異なるだろう。通常、パラグラフCとDは、もし説明が必要であれば詩の実際のあるいは想像上の事情(状況)を示し、そして主題と展開の概略を述べるだろう。もし詩が最初から最後まで第三者の視点による談話であるなら、パラグラフCには行為の簡潔な要約以上のものを含める必要はない。パラグラフDは、主要なアイディアを示し、なぜそれらが他よりも重要なものとされているかを説明するだろう。または物語のどの点が最も強調されているかを示すだろう。
\par
小説は次の見出しの下で議論されうる:
\begin{enumerate}[label=\Alph*.]
    \item Setting.(設定)
    \item Plot.(筋書き)
    \item Characters.(登場人物)
    \item Purpose.(目的)
\end{enumerate}
歴史的な出来事は次の見出しの下で議論されうる:
\begin{enumerate}[label=\Alph*.]
    \item What led up to the event.(その事件の伏線、原因)
    \item Account of the event. (事件の説明)
    \item What the event lead up to.(その事件が誘発したもの) 
\end{enumerate}
最後の2つの主題は、どちらを扱うにしても、書き手はおそらくここに挙げたトピックのひとつかふたつをさらに細分する必要性を感じるだろう。
\par
原則として、単一の文はパラグラフとして書かれたり印刷されたりしてはならない。例外を挙げるなら、説明や議論の部分同士がどういう関係にあるかを示す、変わり目の文だろう。
\par
対話文における一つひとつの発言は、もし単一の語だったとしても、それ自体でパラグラフとなる。つまり、話者が変わるごとに新しいパラグラフが始まる。対話文(dialogue)と物語(narrative)が組み合わさったときにこのルールをどう適用するかは、うまく書かれたフィクションを例にして学ぶのが一番だ。
\section{原則として、パラグラフはトピックセンテンスで始める; 始まりと対応する形で終える}
ここでも、目的は読み手を手助けすることだ。ここで推奨している慣行に従えば、読み手は各パラグラフを読みはじめる時にその趣旨を理解でき、読み終える時にその趣旨を覚えていられる。この理由により、特に説明と議論においては、最も一般的に有用なパラグラフは次のようなものだ。
\begin{enumerate}
    \item トピックセンテンスが冒頭またはその近くにくる;
    \item 続く文が、トピックセンテンスでなされた論述を説明または確立または発展させる;そして
    \item 最後の文は、トピックセンテンスの考えを強調するか、または何か重要な結論を述べる。
\end{enumerate}
脇道にそれて終えたり、つまらないディテールを述べて終えたりすることは、特に避けなければならない。
\par
もしパラグラフがより大きな文章の一部を形作る場合は、前のパラグラフとの関係や、全体の一部として自身が持つ役割を明らかにする必要があるかもしれない。あるときは、これはトピックセンテンス中にちょっとした語や句(again;
therefore; for the same
reason)があれば済んでしまうことがある。しかしながらあるときは、導入や推移のための文を1つないし複数、トピックセンテンスの前に置くほうが都合がいいこともある。もしそのような文が複数必要なら、一般的に言って、推移のための文を別のパラグラフに分けたほうがよい。
\par
書き手の目的によっては、上記のように、書き手はパラグラフの本文をいくつか異なったやり方でトピックセンテンスに関連付けてもよい。書き手は、トピックセンテンスの意味をより明確にするために、それを別の形で再度述べたり、用語を定義したり、逆説を棄却したり、例証したり特定の実例を挙げたりしてもよい;証拠によって立証してもよい;あるいはそれが持つ意味合いと重要性を示して論を展開してもよい。長いパラグラフでは、これらの方法のうちいくつかを実行してもよい。
\begin{enumerate}
    \item 
    \begin{quote}
    Now, to be properly enjoyed, a walking tour should be gone upon alone.
        
        (さて、正しく楽しむためには、徒歩旅行は一人で行かねばならない。)
    \end{quote}
    トピックセンテンス。
    \item 
    \begin{quote}
    If you go in a company, or even in pairs, it is no longer a
walking tour in anything but name; it is something else and more in the
nature of a picnic.

(もし連れだって行くと、たとえ2人で行ったとしても、徒歩旅行というのは名目ばかりのものになってしまう;それは何か別のもので、本質的にはピクニックにより近いものだ。)
    \end{quote}
    逆の説を否定することで、意味がより明確になった。
     \item 
     \begin{quote}
         A walking tour should be gone upon alone, because freedom is of
the essence; because you should be able to stop and go on, and follow
this way or that, as the freak takes you; and because you must have your
own pace, and neither trot alongside a champion walker, nor mince in
time with a girl.

(徒歩旅行は一人で行かなくてはならない。というのも、その本質は自由にあるからだ;気まぐれのままに、立ち止まったり歩きつづけたり、こっちの道を行ったりあっちの道を行ったりできなければならない;そして自分のペースを保たなければならないのだから、ウォーキングのチャンピオンに並んで急ぎ足で歩いたり、女の子に合わせて小股に歩いたりしてはならない。)
     \end{quote}
     トピックセンテンスが手短に繰り返され、3つの理由で支持されている;3番目の理由(自分自身のペースを持たねばならない)の意味は、逆説を否定することにより、いっそう明らかになっている。
     \item 
     \begin{quote}
         And you must be open to all impressions and let your thoughts
take colour from what you see.

(そしてあなたはすべての心象に対して自分を開放し、あなたの思考が目に見えるものから生彩を感じ取るように仕向けなければならない。)
     \end{quote}
     4番目の理由が2つの形で示されている。
     \item 
     \begin{quote}
         You should be as a pipe for any wind to play upon
         
         (あなたはどんな風にも応えて鳴る笛のようでなければならない。)
     \end{quote}
     同じ理由が、また別の形で示されている。
     \item 
     \begin{quote}
         ``I cannot see the wit,'' says Hazlitt, ``of walking and talking
at the same time.
(''そんなことをして何がいいのか私には分からないよ、``とHazlittは言った,''歩きながら話をするなんて。)
     \end{quote}
     \item 
     \begin{quote}
         When I am in the country, I wish to vegetate like the country,"
which is the gist of all that can be said upon the matter.

(田舎にいるときは、私は田舎らしくゆっくりと時間を過ごしたいね、"
この件について一言で言えばそういうことだ。)
     \end{quote}
     6-7 Hazlittが述べたのと同じ理由。
     \item 
     \begin{quote}
         There should be no cackle of voices at your elbow, to jar on the
meditative silence of the morning.
(あなたは、瞑想に誘う朝の静けさを台無しにするようなおしゃべりをしてはいけない。)
     \end{quote}
     Hazlittからの引用を、パラフレーズ(分かりやすくするための言い替え)で繰り返している。
     \item
     \begin{quote}
         And so long as a man is reasoning he cannot surrender himself to
that fine intoxication that comes of much motion in the open air, that
begins in a sort of dazzle and sluggishness of the brain, and ends in a
peace that passes comprehension.--- Stevenson, Walking Tours.

(そして人間に理性がある限り、ある種の眩惑と脳の不活性で始まって理解を超える幸福で終わる、野外で大きなウン動を生み出すあの快適な陶酔に、人は屈することができない。──Stevenson,
Walking Tours.)
     \end{quote}
     4番目の理由について最後の説明を、強い言葉で述べ、力強い結びを形作っている。
\end{enumerate}
\begin{enumerate}
    \item \begin{quote}
        It was chiefly in the eighteenth century that a very different
conception of history grew up.
(歴史に関する非常に特異な考え方が発達したのは主に18世紀においてだった。)
    \end{quote}
    トピックセンテンス。
    \item 
    \begin{quote}
        Historians then came to believe that their task was not so much
to paint a picture as to solve a problem; to explain or illustrate the
successive phases of national growth, prosperity, and adversity.

(歴史学者たちは、彼らの仕事は絵を描くよりもむしろ問題を解決することだと信じるに至った;つまり国家の成長、隆盛、および逆境を、説明または例証することだと。)
    \end{quote}
    トピックセンテンスの意味がより明らかになっている;歴史の新しい考え方が定義されている。
    \item 
    \begin{quote}
        The history of morals, of industry, of intellect, and of art;
the changes that take place in manners or beliefs; the dominant ideas
that prevailed in successive periods; the rise, fall, and modification
of political constitutions; in a word, all the conditions of national
well-being became the subjects of their works.

(倫理、産業、知識、および芸術の歴史;礼儀や信念に生じた変化;続く時代で優勢となった支配的な考え;政体の勃興、凋落、そして変容;一言で言って、国家繁栄のためのすべての条件が、彼らの仕事の主題となった。)
    \end{quote}
    定義が発展している。
    \item 
    \begin{quote}
        They sought rather to write a history of peoples than a history
of kings.

(彼らは王たちの歴史を書くよりは、むしろ人々の歴史を書こうと努めた。)
    \end{quote}
    定義が対比によって説明されている。
    \item 
    \begin{quote}
        They looked especially in history for the chain of causes and
effects.

(彼らは特に歴史の中に、因果関係の鎖を探し求めた。)
    \end{quote}
    定義が補足されている:歴史の新しい考え方における、また別の要素。
    \item \begin{quote}
        They undertook to study in the past the physiology of nations,
and hoped by applying the experimental method on a large scale to deduce
some lessons of real value about the conditions on which the welfare of
society mainly depend. --- Lecky, The Political Value of History.

(彼らは過去に国家の生理学の研究に取りかかったことがある。実験的な方法を大規模な対象に適用して、社会の福祉が主に依存している条件について、現実的な価値を持った何らかの教訓を導き出せないかと期待したのだ。──Lecky,
The Political Value of History.)
    \end{quote}
    結論:歴史の新しい考え方の重要な帰結。
\end{enumerate}
叙述と描写においては、後に続く詳細説明をまとめている簡潔で包括的な記述でパラグラフが始まることが時々ある。
\begin{quote}
The breeze served us admirably.

(風が素晴らしく心地良かった)

The campaign opened with a series of reverses.

(キャンペーンは相次ぐ逆転で幕を明けた)

The next ten or twelve pages were filled with a curious set of
entries.

(続く10-12ページは好奇心をそそるエントリの組でいっぱいだった)
\end{quote}
しかしこの仕掛けは使いすぎるとマンネリズムに陥る。さらによくあることには、最初の文が、そのパラグラフの第一の関心事を主語で示すだけの文になってしまう。
\begin{quote}
At length I thought I might return towards the stockade.

(やっと私は、囲いの中に戻ろうかと考えた)

He picked up the heavy lamp from the table and began to explore.

(彼はテーブルから重いランプを持ち上げて探し始めた)

Another flight of steps, and they emerged on the roof.

(もうひとのぼりして、彼らは屋根の上に出た)
\end{quote}
しかし、生き生きとした物語文の短いパラグラフには、これほどにさえもトピックセンテンスらしさがないことがよくある。そういったパラグラフ間の分断は修辞上の「間」として機能し、行為の細かなディテールを際立たせる。
\section{能動態を使う}
能動態は通常、受動態よりも直接的で力強い:
\begin{quote}
I shall always remember my first visit to Boston.

(初めてBostonを訪れたときのことを私はいつまでも忘れないだろう)
\end{quote}
これは次の文よりもずっと良い。
\begin{quote}
My first visit to Boston will always be remembered by me.

(初めてBostonを訪れたときのことは私によっていつまでも思い出されるだろう)
\end{quote}
後者の文は直接度がより低く、力強さがより低く、簡潔さもより低い。もし書き手が``by
me''(私によって)を割愛することで文をより簡潔にしようとするならば、
\begin{quote}
My first visit to Boston will always be remembered.

(初めてBostonを訪れたときのことはいつまでも思い出されるだろう)
\end{quote}
はっきりしなくなってしまう:訪れたときのことを忘れずにいるのは、書き手なのか、明らかにされていない誰かなのか、それとも世間全般の話なのか?
\par
もちろんこのルールは、書き手が受動態を一切使ってはならないといっているわけではない。受動態は便利なことがよくあるし、ときには必要になる。
\begin{quote}
    The dramatists of the Restoration are little esteemed to-day.
    
    (王政復古の時代の劇作家は、今日ではあまり評価されていない)
    
    Modern readers have little esteem for the dramatists of the
Restoration.

(現代の読者は、王政復古の時代の劇作家をあまり評価していない)
\end{quote}
1番目の文は、王政復古の時代の劇作家たちについてのパラグラフの中であれば正しい形だ;2番目の文は、現代の読者の嗜好についてのパラグラフ中であれば正しい。これらの例のように、特定の語を文の主語にする必要性から、どちらの態を使うべきかが決まることがよくあるだろう。
\par
しかし、習慣的に能動態を使うと力強い文章が生み出される。このことは、主として行為に関する叙述だけの話ではなく、どんな文章においても真実だ。単調な説明文の多くは、there isやcould be heardといったおざなりな表現を能動態の他動詞で置き換えることで、生き生きとして力強い文になる。
\begin{quote}
    There were a great number of dead leaves lying on the ground.
    
    (たくさんの落葉が地面に落ちていた)
    
    Dead leaves covered the ground.
    
    (落葉が地面を覆っていた)
    
    The sound of the falls could still be heard.
    
    (滝の音はまだ聞くことができた)

    The sound of the falls still reached our ears.
    
    (滝の音はまだ私たちの耳に届いた)
    
    The reason that he left college was that his health became
impaired.

(彼が大学を離れた理由は、彼の健康状態が悪化したことだった)

Failing health compelled him to leave college.

(健康上の問題が、大学を去らざるを得ない状況に彼を追い込んだ)

It was not long before he was very sorry that he had said what he
had.

(彼が自分の発言を後悔するまでには、ほとんど時間はかからなかった

He soon repented his words.

(彼はすぐに自分が言ったことを後悔した)
\end{quote}
原則として、他の受動態に直接従属する受動態を作ることは避ける。
\begin{quote}
    Gold was not allowed to be exported.
    
    (金は輸出されることが許されていなかった)
    
    It was forbidden to export gold (The export of gold was
prohibited).

(金を輸出することは禁じられていた(金の輸出は禁じられていた))

He has been proved to have been seen entering the building.

(彼はその建物に入るところを目撃されているということが証明されている)

It has been proved that he was seen to enter the building.

(彼はその建物に入るところを目撃されたことが証明されている)
\end{quote}
上記の例はどちらも、修正する前は、第2の受動態に正しく関係している語は第1の受動態の主語になっている。
\par
よくある失敗は、受動態の構文の主語として、全体の行為を表す名詞を使ってしまい、文を完成させる以外の何の役割も動詞に残してやらないというものだ。
\begin{quote}
    A survey of this region was made in 1900.
    
    (この地域の調査は1990年に行われた)
    
    This region was surveyed in 1900.
    
    (この地域は1900年に調査された)
    
    Mobilization of the army was rapidly carried out.
    
    (軍隊の動員は素早く行われた)
    
    The army was rapidly mobilized.
    
    (軍隊は素早く動員された)
    
    Confirmation of these reports cannot be obtained.
    
    (これらの報告の確認は得られない)
    
    These reports cannot be confirmed.
    
    (これらの報告は確認できない)
\end{quote}
``The export of gold was
prohibited''(金の輸出は禁じられていた)という文を検討してみよう。この文では``was
prohibited''(禁じられていた)という述部が、``export''(輸出)が含意していない何かを表している。
\section{肯定文で記述する}
明確な主張をすること。単調な、はっきりしない、ためらいのある、あいまいな言葉は避ける。not(ない)という語は、否定のために使うかまたはアンチテーゼの中で使い、絶対に言い抜けやはぐらかしの手段として使ってはならない。
\begin{quote}
    He was not very often on time.
    
    (彼が時間を守ることはあまり多くない)
    
    He usually came late.
    
    (彼はたいてい遅れて来る)
    
    He did not think that studying Latin was much use.
    
    (ラテン語を学ぶのがそれほど役に立つとは彼は考えなかった)
    
    He thought the study of Latin useless.
    
    (ラテン語を学んでも役に立たないと彼は考えた)
    
    The Taming of the Shrew is rather weak in spots. Shakespeare does
not portray Katharine as a very admirable character, nor does Bianca
remain long in memory as an important character in Shakespeare's works.

(「じゃじゃ馬馴らし」はところどころに結構な弱点がある。シェイクスピアはキャサリンを非常に尊敬できる人物としては描いておらず、ビアンカもシェイクスピアの作品における重要な人物として長く記憶に残ることがない)

The women in The Taming of the Shrew are unattractive. Katharine
is disagreeable, Bianca insignificant.

(「じゃじゃ馬馴らし」は魅力に欠ける。キャサリンは不愉快だし、ビアンカは影が薄い)
\end{quote}
最後の例の修正前の文は、否定形であると同時にあいまいでもある。したがって修正後の版は書き手の意図を単に推量した。
\par
3つの例はすべてnot(ない)という単語に本質的に存在する弱点を示している。意識的であろうと無意識的であろうと、何がそうではないかということだけしか伝えられないと、読み手は不満に感じる;読み手は何がそうであるかを伝えてほしいのだ。ゆえに一般的には、肯定文で否定を表現するほうがよい。
\begin{quote}
    not honest(正直でない)
    
    dishonest(不正直な)
    
    not important(重要でない)
    
    trifling(些細な)
    
    did not remember(思い出さなかった)
    
    forgot(忘れた)
    
    did not pay any attention to(全く注意を払わなかった)
    
    ignored(無視した)
    
    did not have much confidence in(それほど信用がなかった)
    
    distrusted(不審の念を抱いていた)
\end{quote}
否定と肯定の対照が持つ力は強い:
\begin{quote}
    Not charity, but simple justice.
    
    (慈悲ではなく正義だ)
    
    Not that I loved Caesar less, but Rome the more.
    
    (シーザーを愛していなかった訳ではない、ただローマへの愛のほうが強かったのだ)
\end{quote}
not以外の否定の語はたいてい強力だ:
\begin{quote}
    The sun never sets upon the British flag
    
    (英国旗に陽が沈むことなど決してない)
\end{quote}
\section{不要な語を省く}
力強い文章は簡潔なものだ。文に不要な語が含まれていてはならず、パラグラフに不要な文が含まれていてはならない。それは、絵画に不要な線があってはならず、機械に不要な部品があってはならないのと同じ理由による。これは、書き手がすべての文を短くしなければならないとか、詳細を略して主題の概略だけを扱うべきという訳ではない。そうではなくて、どの単語にも意味がなければならないということだ。
\par
よく使われる多くの表現はこの原則を破っている:
\begin{quote}
    the question as to whether(~かどうかについての質問)
    
    whether (the question whether)(~かどうか(~かどうかという質問)
    
    there is no doubt but that(間違える余地もないことに)
    
    no doubt (doubtless)(間違いなく)
    
    used for fuel purposes(燃料を用途として用いられる)
    
    used for fuel(燃料に使われる)
    
    he is a man who (彼は~というような男だ)
    
    he (彼は)
    
    in a hasty manner(急いだ様子で)
    
    hastily(急いで)
    
    this is a subject which(これは~というような主題だ)
    
    this subject(この主題は)
    
    His story is a strange one.(彼の物語は奇妙なものだった)
    
    His story is strange.(彼の物語は奇妙だった)
\end{quote}
特にthe fact that(~という事実)という表現は、どんな文に出てきたとしても取り除くべきだ。
\begin{quote}
    owing to the fact that(~という事実によって)
    
    since (because)(なぜなら)
    
    in spite of the fact that(~という事実にもかかわらず
    
    though (although)(~なのに)
    
    call your attention to the fact that(~という事実に注意を払ってほしい)
    
    remind you (notify you)(~に注意してほしい)
    
    I was unaware of the fact that(私は~という事実に気がついていなかった)
    
    I was unaware that (did not know)(私は~に気がついていなかった(知らなかった))
    
    the fact that he had not succeeded(彼が成功しなかったという事実)
    
    his failure(彼の失敗)
    
    the fact that I had arrived(私が到着したという事実)
    
    my arrival(私の到着)
\end{quote}
第V章のcase, character, nature, system以下も参照。
\par
Who isやwhich wasやその他同様の語句は、多くの場合余計なものだ。
\begin{quote}
    His brother, who is a member of the same firm(同じ会社の社員であるところの彼の兄弟)
    
    His brother, a member of the same firm(同じ会社の社員である彼の兄弟)
    
    Trafalgar, which was Nelson's last battle(ネルソン提督の最後の戦場であるところのトラファルガー)
    
    Trafalgar, Nelson's last battle(ネルソン提督の最後の戦場トラファルガー)
\end{quote}
肯定形の論述は否定形よりも簡明で、能動態は受動態よりも簡明なので、ルール11と12で示されている例の多くはこのルールの説明にもなっている。
\par
簡潔さに関してよくあるルール違反は、ひとつの複雑な考えを、うまくすればひとつにまとめられるような複数の文で、少しずつ示すことだ。
\begin{quote}
    Macbeth was very ambitious. This led him to wish to become king of
Scotland. The witches told him that this wish of his would come true.
The king of Scotland at this time was Duncan. Encouraged by his wife,
Macbeth murdered Duncan. He was thus enabled to succeed Duncan as king.
(55 words.)

(マクベスは非常に野心的だった。このために彼は、スコットランドの王になることを願うようになった。魔女たちは彼にこの願いはかなうと告げた。時のスコットランドの王はダンカンだった。妻にそそのかされ、マクベスはダンカンを殺害した。彼はこのようにしてダンカンの王位を継承することに成功した。(55語))

Encouraged by his wife, Macbeth achieved his ambition and realized
the prediction of the witches by murdering Duncan and becoming king of
Scotland in his place. (26 words.)

(妻にそそのかされ、ダンカンを殺害してスコットランドの王になることにより、マクベスは自分の野望を達成し、魔女たちの予言を現実のものとした。(26語))
\end{quote}
\section{締まりのない文が連続するのを避ける}
このルールはとりわけ特定の種類の緩い文に当てはまる。2つの対等な節から成っていて、2番目の節が接続詞や関係詞によって導かれるものがそうだ。この種の文は、単独で存在するときは申し分ないかもしれないが(ルール4を参照)、続けて出てくるとすぐに単調で退屈になる。
\par
下手な書き手は往々にして、and, butそして比較的少ないがwho, which, when, whereおよびwhileといった、非制限的な意味の連結語を使って、段落ひとつをまるごとこの種の文で構成してしまうことがあるだろう(ルール3参照)。
\begin{quote}
The third concert of the subscription series was given last
evening, and a large audience was in attendance. Mr.~Edward Appleton was
the soloist, and the Boston Symphony Orchestra furnished the
instrumental music. The former showed himself to be an artist of the
first rank, while the latter proved itself fully deserving of its high
reputation. The interest aroused by the series has been very gratifying
to the Committee, and it is planned to give a similar series annually
hereafter. The fourth concert will be given on Tuesday, May 10, when an
equally attractive programme will be presented.

(一連の番外編コンサートのうちの第3回は昨晩開催され、そしてたくさんの観客が来場していた。エドワード・アップルトン氏が独奏者で、ボストンシンフォニーオーケストラが器楽を担当した。アップルトン氏は自身が第一級の芸術家であることを披露し、一方オーケストラは自身が名声に全く違わぬ実力の持ち主であることを証明してみせた。このシリーズによって生じた利益は委員会にとって満足がいくもので、同様のシリーズを今後も毎年行うことが予定されている。第4回のコンサートは5月10日火曜日に開催される予定で、同様に魅力的なプログラムが上演される)
\end{quote}
陳腐さと内容のなさを別にしたとしても、上記のパラグラフは文の構成が機械的に対称で一本調子なので、出来が悪い。ルール10で引用した文や、適当な出来がいい英語の散文、例えばVanity
Fair(虚栄の市)の序文(Before the
Curtain(開演の前に))と対照させてみてほしい。
\par
もし書き手がこのような種類の文を一続き書いてしまったことに気づいたら、単純な文で置き換えて、単調さがなくなるまで十分に書き直すべきだ。2つの節をセミコロンで接続した文や、2つの節から成る掉尾文や、3つの節から成る緩いもしくは掉尾文調の文か──いずれにせよ考えていることの実際の関係を表すのに最も適した文で置き換える。
\section{対等な複数の考えは同じ形で表現する}
この並列構造の原則によると、似た内容と役割を持った表現は、外見上も似ていなければならない。形の類似のおかげで、内容と役割の類似を読み手が認識しやすい。聖書からの親しみ深い例は、TenCommandments(十戒)、Beatitudes(幸福についての教え(山上の垂訓))、Lord's Prayer(主の祈り)の祈願だ。
\par
下手な書き手は、表現の形式を絶えず変えなければならないという間違った思い込みのせいで、この原則をよく破ってしまう。強調するために記述を繰り返すときは、場合によっては形を変えなければならないというのは真実だ。例えば、ルール10にあるStevensonの作品から引用したパラグラフを見てみてほしい。しかしこのことはさておいて、書き手は並列構造の原則を守るべきだ。
\begin{quote}
    Formerly, science was taught by the textbook method, while now the
laboratory method is employed.

(以前は、科学は教科書を使った方法で教えられていた。一方、今日採用されているのは実験室を使った方法だ)

Formerly, science was taught by the textbook method; now it is taught by the laboratory method.

(以前は、科学は教科書を使った方法で教えられていた;今日では科学は実験室を使った手法で教えられている)
\end{quote}
左側の例は、書き手に決断力がないか自信に欠けているという印象を与える;書き手はひとつの表現形式を選んで決めて踏み止まることができないか、またはそうすることを恐がっているように見える。右側の例は少なくとも書き手が選択肢を選び、それを守っていることを示している。
\par この原則により、一連の要素すべてに適用される冠詞や前置詞は、最初の語の前だけで使うか、そうでなければ各要素の前で繰り返さなければならない。
\begin{quote}
    The French, the Italians, Spanish, and Portuguese
    
    The French, the Italians, the Spanish, and the Portuguese
    
    In spring, summer, or in winter
    
    In spring, summer, or winter (In spring, in summer, or in winter)
\end{quote}
相関的な表現(both, and; not, but; not only, but also; either, or;
first, second,
thirdおよびその他類似したもの)は、文法上同じ構成が後に続かなければならない。このルールを破っている例の多くは、文を整理し直すことで修正できる。
\begin{quote}
    It was both a long ceremony and very tedious.
    
    (長い式典でとても退屈なことの両方だった)
    
    The ceremony was both long and tedious.
    
    (式典は長くしかも退屈だった)
    
    A time not for words, but action
    
    (言葉の時ではなく、行動だ)
    
    A time not for words, but for action
    
    (言葉の時ではなく、行動の時だ)
    
    Either you must grant his request or incur his ill will.
    
    (彼の要求を入れなければならないか、彼の不興を買わなければならない)
    
    You must either grant his request or incur his ill will.
    
    (彼の要求を入れるか、彼の不興を買うかしなければならない)
    
    My objections are, first, the injustice of the measure; second,
that it is unconstitutional.

(私の反論は、第一に、基準の不公正さだ;第二に、それが憲法違反であることだ)

My objections are, first, that the measure is unjust; second, that
it is unconstitutional.

(私の反論は、第一に、基準が不公正であること;第二に、それが憲法違反であることだ)
\end{quote}
ルール12の3番目の例と、ルール13の最後の例も参照。
\par 書き手が非常に多くの似たような考えを表現しなければならないとしたらどうするのか、という疑問が生じるかもしれない。例えば20もあったらどうするのか?
書き手は同じパターンの文を20も続けて書かなければならないのだろうか?
よく調べてみれば、この問題は杞憂であり、20のアイデアはグループに分けられ、原則を適用するのはグループ内だけでいいことが分かるだろう。さもなければ、記述を表の形式にすることで問題を回避するのが最善の手だ。
\section{関係する語句は一緒にしておく}
文中の語の位置は、語の関係を示すための第一の手段だ。したがって書き手は、可能な限り、内容において関係する語および語のグループはまとめなければならず、それほど関係が深くないものは離しておかなければならない。
\par
文の主語と主たる動詞とは、原則として、文頭に移動可能な句や節で分断されてはならない。
\begin{quote}
    Wordsworth, in the fifth book of The Excursion, gives a minute
description of this church.

(ワーズワースは、The
Excursionの第5巻目で、この教会を詳細に描写している)

In the fifth book of The Excursion, Wordsworth gives a minute
description of this church.

(The Excursionの第5巻目で、ワーズワースはこの教会を詳細に描写している)

Cast iron, when treated in a Bessemer converter, is changed into
steel.

(鋳鉄は、ベッセマー転炉で処理すると、鋼鉄に変化する)

By treatment in a Bessemer converter, cast iron is changed into
steel.

(ベッセマー転炉で処理することで、鋳鉄は鋼鉄に変化する)
\end{quote}
分断に反対する理由は、挿入された句や節が、主節の自然な順序をいたずらにさえぎるということだ。しかしながらこの反対理由は、順序が単に関係節によって割り込まれたり同格の表現によって割り込まれたりした場合には、主張できないのが普通だ。割り込みが不安定さを作り出すための手段として意図的に使われる掉尾文においても、これは成り立たない(ルール18を参照)
\par 関係代名詞は原則としてその先行詞の直後にこなければならない。
\begin{quote}
    There was a look in his eye that boded mischief.
    
    (いたずらの予兆が彼の目の中に見て取れた)
    
    In his eye was a look that boded mischief.
    
    (彼の目の中にはいたずらの予兆が見て取れた)
    
    He wrote three articles about his adventures in Spain, which were
published in Harper's Magazine.

(彼はスペインでの彼の冒険に関して3件の記事を書き、それらはHarper's
Magazineで発表された)

He published in Harper's Magazine three articles about his
adventures in Spain.

(彼はスペインでの彼の冒険に関して3件の記事をHarper's
Magazineで発表した)

This is a portrait of Benjamin Harrison, grandson of William Henry
Harrison, who became President in 1889.

(これはBenjamin
Harrisonの肖像画だ。彼はWilliam Henry
Harrisonの孫息子で、1889年に大統領になった)

This is a portrait of Benjamin Harrison, grandson of William Henry
Harrison. He became President in 1889.

(これはWilliam Henry
Harrisonの孫息子であるBenjamin
Harrisonの肖像画だ。彼は1889年に大統領になった)
\end{quote}
もし先行詞が語のグループから成る場合は、そうすることで意味があいまいにならない限り、関係詞はグループの末尾にくる。
\begin{quote}
    The Superintendent of the Chicago Division, who
    
    (シカゴ地区本部長)
    
    A proposal to amend the Sherman Act, which has been variously
judged

(シャーマン法を改正する、多角的に審議された提案が)

A proposal, which has been variously judged, to amend the Sherman
Act

(多角的に審議された、シャーマン法を改正するための提案が)

The grandson of William Henry Harrison, who

(William Henry Harrisonの孫息子)

William Henry Harrison's grandson, Benjamin Harrison, who

(William Henry Harrisonの孫息子Benjamin Harrison)
\end{quote}
同格の名詞は先行詞と関係詞の間にあってもよい。というのも、そのような組み合わせでは実際にはあいまいさが生じえないからだ。
\begin{quote}
    The Duke of York, his brother, who was regarded with hostility by
the Whigs

(彼の兄弟であり、ホイッグ党から敵対的にみられているヨーク公)
\end{quote}
修飾語は、もし可能なら修飾の対象となる語に隣接して置かなければならない。もしいくつかの表現が同一の語を修飾する場合、間違った関連を連想させないように配置しなければならない。
\begin{quote}
    All the members were not present.
    
    (メンバー全員が出席していなかった)
    
    Not all the members were present.
    
    (メンバー全員が出席していた訳ではなかった)
    
    He only found two mistakes.
    
    (彼は2つの誤りを見つけただけだった)
    
    
    He found only two mistakes.
    
    (彼はたった2つの誤りしか見つけなかった)
    
    Major R. E. Joyce will give a lecture on Tuesday evening in Bailey
Hall, to which the public is invited, on ``My Experiences in
Mesopotamia'' at eight P.M.

(R. E. Joyce少佐が火曜日の晩にBailey
Hallでレクチャーを行う。レクチャーは一般参加可能で、「メソポタミアにおける私の経験」というテーマで、午後8時開始)

On Tuesday evening at eight P.M., Major R. E. Joyce will give in
Bailey Hall a lecture on ``My Experiences in Mesopotamia.'' The public
is invited.

(火曜日の晩、午後8時に、R. E. Joyce少佐がBailey
Hallで、「メソポタミアにおける私の経験」というテーマでレクチャーを行う。レクチャーは一般参加可能)
\end{quote}
\section{サマリでは時制をひとつだけに}
戯曲の筋を要約する際には、書き手は常に現在形を使わなければならない。詩、物語、または小説を要約する際には、書き手の好みによっては過去形を使ってもよいが、現在形を使うことが望ましい。もし要約が現在形になっている場合、先立つ行動は完了時制で表現しなければならない。過去形になっている場合は過去完了で表現する。
\begin{quote}
    An unforeseen chance prevents Friar John from delivering Friar
Lawrence's letter to Romeo. Juliet, meanwhile, owing to her father's
arbitrary change of the day set for her wedding, has been compelled to
drink the potion on Tuesday night, with the result that Balthasar
informs Romeo of her supposed death before Friar Lawrence learns of the
nondelivery of the letter.

(予想外の巡り合わせのために、Friar
JohnはFriar
Lawrenceの手紙をRomeoに届けられない。そのころJulietは、彼女の父が気まぐれで彼女の結婚式の日を変えたおかげで、火曜の夜には薬を飲み終えていた。その結果、Friar
Lawrenceが手紙の不達を知る前に、BalthasarはRomeoに彼女の「死」を伝える)
\end{quote}
しかし要約でどの時制が使われようと、間接話法や間接疑問における過去形は変わらない。
\begin{quote}
    The Legate inquires who struck the blow.
    (使節は一撃を加えたのが誰かを問うた)
\end{quote}
注記した例外を除けば、どの時制を書き手が選んだとしても、貫徹しなければならない。ある時制から別の時制へと移り変わると、不明確で優柔不断な見た目を与える(ルール15と比較せよ)。
\par
エッセイを要約するときやスピーチを報告するときのように、誰か他の人の言説や考えを示すときには、書き手は``he
said,''(彼は言った)、``he stated,''(彼は述べた)、``the speaker
added,''(話し手は付け加えた)、``the speaker then went on to
say,''(話し手はこう続けた)、``the author also
thinks,''(筆者はまたこう考える)のような表現を差し挟んではならない。続くものが要約であることを、書き手は最初からきっぱりと明確に示さなければならず、そしてその注意を繰り返して無駄に言葉を費してはならない。
\par
手帳や新聞や文学作品の解説書などにおいて、何らかの要約は必須であり、そして小学校に通う子供たちにとっては物語を自分の言葉で語り直すことは有用な訓練になる。しかし批評や文学作品の解釈においては、書き手は要約に没頭しないように注意しなければならない。書き手は1、2文を費して、自分が論じている作品の主題や冒頭の状況を示す必要があると考えるかもしれない;詳細を数多く引用して作品の出来を例証してもよいだろう。しかし書き手は、要約に時折コメントが付くようなものではなく、証拠に基づいている整然とした論考を書くよう心がけなければならない。同様に、書き手が論じる範囲にいくつもの作品が含まれていたら、原則として書き手は時系列順に単調に取り上げるのではなく、最初から総合的な結論を固めることに狙いを定めたほうがよい。
\section{強調する語は一文の中で最後に置く}
書き手が最も目立たせたいと思う語や語のグループを置くのに適切な位置は、通常は文の末尾だ。
\begin{quote}
    Humanity has hardly advanced in fortitude since that time, though
it has advanced in many other ways.

(人間性は、忍耐力に関してはそのころからほとんど進歩がない──他の多くの面では進歩したとはいえ。

Humanity, since that time, has advanced in many other ways, but it
has hardly advanced in fortitude.

(人間性はそのころから、他の多くの面では進歩してきた、しかし忍耐力に関してはほとんど進歩していない)

This steel is principally used for making razors, because of its
hardness.

(この鋼は主として剃刀を作るのに使われる。その理由は硬さにある)

Because of its hardness, this steel is principally used in making
razors.

(その硬さのために、この鋼は主として剃刀を作るのに使われる
\end{quote}
この目立つ場所にふさわしい語または語のグループは、通常は論理的な述語だ。つまり、2番目の例に見られるように文中の新しい要素だ。
\par
掉尾文が効果的なのは、それが主たる論述を際立たせるからだ。
\begin{quote}
    Four centuries ago, Christopher Columbus, one of the Italian
mariners whom the decline of their own republics had put at the service
of the world and of adventure, seeking for Spain a westward passage to
the Indies as a set-off against the achievements of Portuguese
discoverers, lighted on America.

(4世紀前、イタリア海軍の一員で、自分たちの共和国が凋落したことが理由となって、ポルトガルの探険家たちの手柄に見合う成果として、スペインのためにインドへの西向き航路を探し出そうという世界的な冒険の任務にあたったChristopher
Columbusは、アメリカを見つけた)

With these hopes and in this belief I would urge you, laying aside
all hindrance, thrusting away all private aims, to devote yourselves
unswervingly and unflinchingly to the vigorous and successful
prosecution of this war.

(これらの希望とこの信念をよりどころに、私はあなたに強く提案する。すべての邪魔を脇に置いて、私的な目標を横へ押しやり、そしてこの戦争に対する精力的で効果的な糾弾に断固として自身を委ねることを)
\end{quote}
文中でもうひとつ目立つ場所は文頭だ。文の要素はいずれも、主語を除いて、文頭に置くと強調される。
\begin{quote}
    Deceit or treachery he could never forgive.
    
    (欺瞞や裏切りを、彼は絶対に許せなかった。)
    
    So vast and rude, fretted by the action of nearly three thousand
years, the fragments of this architecture may often seem, at first
sight, like works of nature.

(非常に広範にわたって荒々しく、ほぼ3000年近くの環境作用によって侵食されているので、この建築物の一部分は多くの場合、一見自然の産物のように見えるだろう)
\end{quote}
主語がその文の最初にくる場合は、強調されるかもしれないが、位置を変えただけではほとんど効果はない。次の文では、
\begin{quote}
    Great kings worshipped at his shrine,
    
    (偉大な王たちは彼の神殿に参拝した)
\end{quote}
kingsが強調される効果は、大半がその意味と文脈によって生じている。特別に強調されるためには、文の主語は述部の位置になければならない。
\begin{quote}
    Through the middle of the valley flowed a winding stream.
    
    (谷の中央を貫いて、曲がりくねった小川が流れていた)
\end{quote}
最も目立たせるべきものの適切な位置が末尾であるという原則は、文中の語にも、パラグラフ中の文にも、文章中のパラグラフにも同様に当てはまる。
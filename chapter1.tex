\chapter{序章}
この本は、文章構成の練習と文学の研究とを兼ねた英語のコースで使われることを想定している。この本では、平易な英語のスタイルに関する一番大事な必要条件を、手短に示すつもりだ。この本では、いくつかの本質的要素、つまり最も間違えられることが多い語法のルールおよび文章構成の原則とに(第II章と第III章で)注意力を集中することにより、講師と生徒の負荷を軽減することをねらっている。セクションの番号は原稿を修正する際のリファレンスとして役に立つだろう。
\par
この本は英語のスタイルという分野のほんのわずかの部分をカバーするに過ぎない。だが筆者の経験では、いったん本質的な要素をものにしてしまえば、生徒にとって一番効果が高いのは自分が書いた文章の問題点に基づいて個別指導を受けることであり、講師はみなそれぞれ自分なりの理論体系を持っていて、どんな教科書で示されたものよりも自分の理論体系を好むものだ。
\par
Cornell大学の英語部門における筆者の同僚たちは、原稿を準備するにあたって多大な協力をしてくれた。George
McLane Wood氏は親切にも、彼の「書き手への提言」(Suggestions to
Authors.)に由来する材料の一部分をルール11の配下に含めることに同意してくれた。
\par
次の書籍を、さらに進んで学ぶ際の資料として推奨する:
第II章と第IV章に関連して、F. Howard Collins, Author and Printer (Henry
Frowde); Chicago University Press, Manual of Style; T. L. De Vinne
Correct Composition (The Century Company); Horace Hart, Rules for
Compositors and Printers (Oxford University Press); George McLane Wood,
Extracts from the Style-Book of the Government Printing Office (United
States Geological Survey); 第III章と第V章に関連して、Sir Arthur
Quiller-Couch, The Art of Writing (Putnams), 特にInterlude on
Jargonの章; George McLane Wood, Suggestions to Authors (United States
Geological Survey); John Leslie Hall, English Usage (Scott, Foresman and
Co.); James P. Kelly, Workmanship in Words (Little, Brown and Co.).
\par
この上なく優れた書き手は時にレトリックのルールを破ることがあると、古くから知られている。しかしながら彼らがそうするときはたいてい、その文にはルールを破るに値するメリットがあることが、読み手にも分かるものだ。同様にうまく書けるという確信がない限り、おそらく原則を守るのが最善だろう。優れた書き手の指導により、日常利用に適した簡素な英語を書くことを学んだその後で、スタイルの秘密を探求させ、文学の達人たちを研究することへと目を向けさせよう。
\chapter{基本的な用法の原則}
\section{名詞の所有格単数を'sで作る}
最後の子音が何であれこのルールを守ること。したがって次のように書く。
\begin{quote}
Charles's friend

Burns's poems

the witch's malice    
\end{quote}
これがUnited States Government Printing OfficeおよびOxford University Pressでの用法である。
\par
例外は、古来の固有名詞で-esと-isで終わるものの場合だ。例えば所有格のJesus'や、for
conscience' sake, for righteousness' sakeといった形式だ。しかしAchilles'
heel, Moses' laws, Isis'
templeといった形式は、たいてい次のように置き換えられる。
\begin{quote}
the heel of Achilles

the laws of Moses

the temple of Isis    
\end{quote}
代名詞の所有格であるhers, its, theirs, yoursおよびoneselfにはアポストロフィを付けない。
\section{単独の接続詞の後に3つかそれ以上の語句が連続する場合、末尾を除く各語句の後にカンマを置く}
したがってこう書く。
\begin{quote}
red, white, and blue

honest, energetic, but headstrong

He opened the letter, read it, and made a note of its contents.    
\end{quote}
これもGovernment Printing OfficeおよびOxford University
Pressでの用法である。
\par
企業の名称では、最後のカンマは次のように省略される。
\begin{quote}
Brown, Shipley and Company    
\end{quote}
etc.という略語は、もしもその前にあるのが単独の語句だったとしても、常に直前にカンマを付ける。
\section{カンマの間に挿入句的な表現を挿入する}
\begin{quote}
The best way to see a country, unless you are pressed for time, is to
travel on foot.    
\end{quote}
このルールは適用するのが難しい。howeverのように単独の語や短い語句が、挿入句(parenthetic)なのかどうかは、見極めにくいことがよくある。もし文の流れにわずかしか割り込まないのであれば、カンマを省略しても問題ない。しかし割り込みがわずかであろうと相当であろうと、カンマを1つだけ略して他をそのままにするということは許されない。
\begin{quote}
Marjorie's husband, Colonel Nelson paid us a visit yesterday,    
\end{quote}
あるいは
\begin{quote}
My brother you will be pleased to hear, is now in perfect health,    
\end{quote}
のような句読点の打ち方は、弁護の余地なく誤りだ。
\par
非制限的関係節(Non-restrictive relative
clauses)は、このルールに従ってカンマで区切られる。
\begin{quote}
The audience, which had at first been indifferent, became more and more
interested.    
\end{quote}
whereおよびwhenによって導入される同様の節も、同様に句読点を打つ。
\begin{quote}
In 1769, when Napoleon was born, Corsica had but recently been acquired
by France.

Nether Stowey, where Coleridge wrote The Rime of the Ancient Mariner, is
a few miles from Bridgewater.    
\end{quote}
これらの文において、which,
whenおよびwhereによって導かれる節は非制限的だ。これらの節は従属している語の適用を制限せずに、挿入句的に平叙文(statements)を付け加えて主部を補う。各文はそれぞれ独立可能な2つの平叙文の組み合わせから成っている。
\begin{quote}
The audience was at first indifferent. Later it became more and more
interested.

Napoleon was born in 1769. At that time Corsica had but recently been
acquired by France.

Coleridge wrote The Rime of the Ancient Mariner at Nether Stowey. Nether
Stowey is only a few miles from Bridgewater.    
\end{quote}
制限的な関係節(Restrictive relative
clauses)はカンマで区切らない。
\begin{quote}
The candidate who best meets these requirements will obtain the place.    
\end{quote}
この文では、単独の人物に対するcandidateという語の適用を、関係節が制限している。これまでの例とは異なり、この文は2つの独立した平叙文に分離できない。
\par
etc.あるいはjr.という略称は、常に直前にカンマがくる。そして文末にあるときを除いて常に直後にカンマがくる。
\par
挿入句的表現をカンマで囲むのと基本的には同様に、文の主部のすぐ前あるいはすぐ後にくる句または従属節は、カンマで区切る。このセクションのルール4,
5, 6, 7, 16および18で引用された文が十分な説明となるだろう。
\par
もし挿入句的な表現の前に接続詞が置かれているなら、最初のカンマを接続詞の後ではなく前に置くこと。
\begin{quote}
He saw us coming, and unaware that we had learned of his treachery,
greeted us with a smile.    
\end{quote}
\section{独立した節を導入するandやbutの前にはカンマを置く}
\begin{quote}
The early records of the city have disappeared, and the story of its
first years can no longer be reconstructed.

The situation is perilous, but there is still one chance of escape.    
\end{quote}
この種の文は、文脈を別にすれば、書き直す必要があるかのように見えるかもしれない。これらの文はカンマに達したときに完全に意味を成すので、2番目の節は追加表現のように見える。そのうえ、andは接続詞のなかで最もあいまいなものだ。独立した節の間で使われた場合、2つの節に何らかの関係があることを示すが、その関係がどのようなものであるかをはっきり示さない。上記の例では、両者の関係は原因と結果だ。2つの文は次のように書き換えられる:
\begin{quote}
As the early records of the city have disappeared, the story of its
first years can no longer be reconstructed.

Although the situation is perilous, there is still one chance of escape.    
\end{quote}
もしくは、従属節を句で置き換えてもよい:
\begin{quote}
Owing to the disappearance of the early records of the city, the story
of its first years can no longer be reconstructed.

In this perilous situation, there is still one chance of escape.    
\end{quote}
しかし書き手は、文をあまりに満遍なく短く掉尾文調(periodic)にしてしまうという間違いを犯すかもしれないし、ところどころに緩い文があると文体が極端に堅苦しくはならず、読み手はそれなりの息抜きができる。よって、最初に引用したような緩い文は、気楽で自然体な書き物でよくみられる。しかし書き手は、あまり多くの文をこのパターンで書いてしまわないよう、注意しなければならない(ルール14を参照)。
\par
2つの部分から成り、2番目の部分が(becauseの意の)as, for, or,
norおよび(and at the same
timeの意の)whileで始まる文も同様に、接続詞の前にカンマが必要となる。
\par
もし、従属節またはカンマで区切られるべき導入句が、2番目の独立した節(主節)の前にある場合は、接続詞の後にカンマは必要ない。
\begin{quote}
The situation is perilous, but if we are prepared to act promptly, there
is still one chance of escape.
\end{quote}
副詞で接続された、2つの部分から成る文については、次のセクションを参照。
\section{独立した節をカンマで接続してはならない}
もし、文法上完全でかつ接続詞で接続されていない2つ以上の節が、単一の複合文を構成するときは、句読点として正しい記号はセミコロンだ。
\begin{quote}
Stevenson's romances are entertaining; they are full of exciting
adventures.

It is nearly half past five; we cannot reach town before dark.    
\end{quote}
もちろん上記を、セミコロンをピリオドに置き換えてそれぞれ2つの文として書くのも正しい。
\begin{quote}
Stevenson's romances are entertaining. They are full of exciting adventures.

It is nearly half past five. We cannot reach town before dark.
\end{quote}
もし接続詞が挿入されたなら、適切な記号はカンマとなる(ルール4)。
\begin{quote}
Stevenson's romances are entertaining, for they are full of exciting
adventures.

It is nearly half past five, and we cannot reach town before dark.    
\end{quote}
ここで注意。もし2番目の節の前に、接続詞ではなく、accordingly, besides, so, then, thereforeまたはthusといった副詞がきていたら、やはりセミコロンが必要になる。
\begin{quote}
I had never been in the place before; so I had difficulty in finding my
way about.    
\end{quote}
しかしながら一般的には、ライティングにおいてはこのようなやり方でsoを使うことは避けるのが最善だ。危険なことに、それを少しでも使う書き手は、使いすぎの気がある。これを直すための、簡単でたいてい役に立つ方法は、soという単語を使わずに、最初の節をasで始めることだ:
\begin{quote}
As I had never been in the place before, I had difficulty in finding my
way about.    
\end{quote}
もし節がとても短くて、互いに形式が似ていたら、カンマでも通常は差し支えない:
\begin{quote}
Man proposes, God disposes.

The gate swung apart, the bridge fell, the portcullis was drawn up.    
\end{quote}
\section{文を2つに分割してはならない}
別の言い方をすれば、カンマの代わりにピリオドを使ってはならない。
\begin{quote}
I met them on a Cunard liner several years ago. Coming home from
Liverpool to New York.

He was an interesting talker. A man who had traveled all over the world,
and lived in half a dozen countries.
\end{quote}
これらの例のどちらにおいても、最初のピリオドはカンマで置き換えられるべきで、続く語は小文字で始まるべきだ。
\par
文の目的に合わせて語や表現を強調し、そのために句読点を打つことは許容される。
\begin{quote}
Again and again he called out. No reply.    
\end{quote}
しかしながら書き手は、その強調が正当なものであることに、そして単に句読点の打ち方を間違えているのではないかと疑われないことに、確信がなければならない。
\par 
ルール3, 4,
5および6は、通常の文の句読法における最も重要な原則をカバーする。これらの原則は完全にマスターし、第二の天性のように習慣として使いこなせなければならない。
\section{文頭の分詞句は文法上の主語を受けていなければならない}
\begin{quote}
Walking slowly down the road, he saw a woman accompanied by two children.    
\end{quote}
walkingという語は文の主語を受けており、womanを受けてはいない。もし書き手がそれでwomanを受けるようにしたいのであれば、文を書き直さなければならない:
\begin{quote}
    He saw a woman, accompanied by two children, walking slowly down the road.
\end{quote}
接続詞または前置詞が前にきている分詞句、同格の名詞、形容詞、形容詞句は、それらが文頭にある場合、同じルールに従う。
\begin{quote}
On arriving in Chicago, his friends met him at the station.

When he arrived (or, On his arrival) in Chicago, his friends met
him at the station.

A soldier of proved valor, they entrusted him with the defence of
the city.

A soldier of proved valor, he was entrusted with the defence of the city.

Young and inexperienced, the task seemed easy to me.

Young and inexperienced, I thought the task easy.

Without a friend to counsel him, the temptation proved irresistible.

Without a friend to counsel him, he found the temptation
irresistible.
\end{quote}
このルールに違反した文は、たいてい滑稽なものだ。
\begin{quote}
Being in a dilapidated condition, I was able to buy the house very cheap.    
\end{quote}
\section{語の構成と発音に従い、行末で語を分割する}
もし行末に、語のうちの1音節以上が入る余裕があるが、語全体が入るほどの余裕はない場合、語を分割する。ただし1文字だけを切り落としたり、長い語の2文字だけを切り落としたりするようなことになってはいけない。どんな語にも適用できる、確実で手間いらずのルールを定めることはできない。適用できる場合が最も多い原則は次のものだ:
\begin{enumerate}
    \item 語の構成によって分割する:
    \begin{quote}
        know-ledge (not knowl-edge); Shake-speare (not Shakes-peare); de-scribe (not des-cribe); atmo-sphere (not atmos-phere)
    \end{quote}
    \item 「母音の上で」分割する:
     \begin{quote}
        edi-ble (not ed-ible); propo-sition; ordi-nary; espe-cial; reli-gious; oppo-nents; regu-lar; classi-fi-ca-tion (three divisions possible); deco-rative; presi-dent;
    \end{quote}
    \item 二重になった文字の間で分割する、ただし語中の単純な構造の最後にくる場合を除く:
     \begin{quote}
        Apen-nines; Cincin-nati; refer-ring; but tell-ing.
    \end{quote}
\end{enumerate}
組み合わさった子音の扱いは、例を見るのが一番だろう:
\begin{quote}
for-tune; pic-ture; presump-tuous; illus-tration; sub-stan-tial (either
division); indus-try; instruc-tion; sug-ges-tion; incen-diary.    
\end{quote}
丁寧に印刷された本の何ページにもわたって音節の分割を調べるという仕事を、学生は見事にこなすだろう。

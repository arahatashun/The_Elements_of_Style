\chapter{単語と表現でよくある間違い}
(ここに列挙した語と表現の多くは、そのスタイルのひどさに比べれば、英語としてはそれほどひどいものではない。これらのスタイルは、ずさんなライティングのありきたりな例だ。Featureで説明されているように、適切な修正を施すためには、単語を取り換えるのではなく、あいまいで漠然とした記述を明確な記述で置き換える必要があるだろう。)
\begin{description}
\item [All right(よろしい)]
「同意した」あるいは「そうしなさい」という意味の独立したフレーズとして、日常口語で使われるのが慣用。他の使い方は避けること。常に2つの単語として表記される。
\item[As good or better than(同じかそれよりもよい)]
この種の表現は文を整理し直して修正すべきだ。
\begin{quote}
    My opinion is as good or better than his.
    
    (私の意見は彼の意見と同じかそれ以上に優れている)
    
    My opinion is as good as his, or better (if not better).
    
    (私の意見は彼の意見と同じくらい優れている。あるいはそれ以上に優れている(それ以上に優れていないとしても)
\end{quote}
\item [ As to whether(かどうか)]Whetherで十分だ;ルール13を参照。
\item[Bid(入札する)]toがない不定詞をとる。過去形はbade.
\item [Case(ケース)] The Concise Oxford
Dictionaryはその定義をこの言葉で始めている:``instance of a thing's
occurring; usual state of
affairs.''(物事の発生の事例;事物のいつもの状態)
これらの2つの意味において、この語は通常不要だ。
\begin{quote}
    In many cases, the rooms were poorly ventilated.
    
    (多くの場合、それらの部屋はよく換気されていなかった)
    
    Many of the rooms were poorly ventilated
    
    (それらの部屋の多くはよく換気されていなかった)
    
    It has rarely been the case that any mistake has been made.
    
    (何か間違いをするというケースは、ほとんどなかった)
    
    Few mistakes have been made.(間違いはほとんどなかった)
\end{quote}
Wood, Suggestions to Authors, pp.~68-71, およびQuiller-Couch, The
Art of Writing, pp.~103-106を参照。
\item [Certainly(確かに)]
veryを見境なしに使う者がいるのと同様に、どんなものであろうとすべての単語に対して強調のためにこの言葉を使う者がいる。この種のマンネリズムはスピーチではまずいものだし、ライティングにおいてはなおさらまずい。
\item [Character(性格)]
単にくどい話し方が癖になって使われているだけで、全く冗長でしかない場合が多い。
\begin{quote}
    Acts of a hostile character(敵対的な性格を示す行為)
    
    Hostile acts(敵対的な行為)
\end{quote}
\item[Claim, vb.(主張)]目的語名詞を伴うと、lay claim
to(主張する)という意味になる。この意味を明確に含んでいるならば、従属節と共に使ってもよい:``He
claimed that he was the sole surviving
heir.''(彼は自分がたった一人の生き残った相続人だったと主張した)(ただしここでさえ、``claimed
to be''(であると主張)のほうが望ましいだろう。)declare,
maintain,またはchargeの代替として使われるべきものではない。
\item [Compare(比較する)]compare
toとは、本質的に異なる種類の物事の間の類似を指摘または示唆することだ;compare
withとは、主に本質的に同じ種類の物事の間の違いを指摘することだ。かくして人生は巡礼や劇や戦いに比較(compare
to)されてきた;米国議会は英国議会と比較(compare
with)してもよいだろう。パリは古代アテネに比べられてきた(compare
to);現代のロンドンと比較(compare with)してもよいだろう。
\item [Clever(賢しい)]
この言葉はあまりにも使われすぎている;小さなことについて発揮された知恵に限って使うようにするのが最善だ。
\item[Consider(考慮する)]``believe to be.''(みなす) ``I consider him thoroughly
competent.''(彼は全く適任であると私は考えている)という意味のときにはasが後に続かない。``The
lecturer considered Cromwell first as soldier and second as
administrator''(講師はクロムウェルを第一に兵士、第二に行政官とみなしていた)と比較せよ。ここでは``considered''は``examined''(検討した)または``discussed''(考察した)という意味だ。
\item [Dependable(信頼できる)]reliable, trustworthyの不要な言い替え。
\item [Due to(のせいで)] 副詞句においてthrough, because of,またはowing
toの代わりに不適切に使われる:``He lost the first game, due to
carelessness.''(彼は最初のゲームで不注意のせいで負けた)
述語として、または特定の名詞に関する修飾語として関連した正しい用法では:``This
invention is due to Edison;''(この発明はエジソンによる) ``losses due
to preventable fires''(防ぐことができた火事による損失)。
\item [Effect(効果)]
名詞としては、result(結果)を意味する;動詞としては、to bring
about,
accomplish(もたらす、達成する)を意味する(affectと混同してはならない。それは``to
influence''(影響を及ぼす)という意味)。
\par 
名詞としては、ファッションや音楽や絵画その他の芸術に関するいい加減なライティングで、ゆるく使われることがよくある:``an
Oriental effect;''(東洋風の効果) ``effects in pale
green;''(薄緑の効果) ``very delicate effects;''(とても微妙な効果)
``broad effects;''(幅広い影響) ``subtle effects;''(かすかな効果) ``a
charming effect was produced
by''(魅力的な効果はそれによって生み出されている)。伝えるべき明確な趣旨を持った書き手は、そのようなあいまいな表現に逃げたりしないだろう。
\item [Etc(など)]人間に対して使ってはならない。これはand the rest, and so
forth,と同等の表現で、そのためこれらの表現のいずれかを使うのが不適当な場合、つまり、もし読み手が何か大事な物事の詳細が分からないまま置きざりにされるようなら、使ってはならない。それが既に完全な形で与えられたリストの最後の語か、または引用文の末尾の重要でない語を表しているのであれば、etc.を使うことに異論を挟む余地はほとんどない。
\par
such asやfor
exampleやその他似たような表現によって導かれるリストの末尾では、etc.を使うのは正しくない。
\item [Fact(事実)]この語は判断に関する場合には使うべきではなく、直接証明が可能な場合に限って使うこと。特定の出来事がある日時に起きたということや、鉛が特定の温度で溶けるということは、事実(fact)だ。しかし、ナポレオンが近代の将軍のなかで最も優れている、またはカリフォルニアの気候は快適であるというような判断は、それらがいかに明白であろうと、正しい事実とはいえない。
\par the fact that,という定型については、ルール13を参照。
\item[Factor(要因)]
紋切型で陳腐な言葉だ;この言葉を含む表現は、たいていより直接的でふさわしい表現で置き換えることができる。
\begin{quote}
    His superior training was the great factor in his winning the
match.

(彼のより優れた訓練が、彼が試合に勝利する大きな要因だった)

He won the match by being better trained.

(彼はより良く訓練されていたので試合に勝利した)

Heavy artillery is becoming an increasingly important factor in
deciding battles.

(戦闘の勝敗の決め手となる要因として、重砲はますます重要になってきている)

Heavy artillery is playing a larger and larger part in deciding
battles.

(戦闘の勝敗の決め手として、重砲はますます大きな役割を担うようになっている)
\end{quote}
\item [Feature(特徴)] また別の紋切型で陳腐な語;factorと同じように、それが使われている文に何の意味も加えないことが多い。
\begin{quote}
    A feature of the entertainment especially worthy of mention was
the singing of Miss A.

(その演目の特徴で特に言及に値するのは、A嬢の歌唱だ)

\end{quote}
同じだけの語数を費すなら、何をA嬢は歌ったのか、またはもしプログラムが既に提供されているのであれば、彼女がどのように歌ったかについて述べたほうがよい
\par 動詞としては、offer as a special attraction,(特別魅力的なオファー)という広告的な意味では、避けるべきだ。
\item [Fix(直す)]アメリカ口語ではarrange, prepare,
mend.という意味がある。ライティングにおいてはfasten, make
firmまたはimmovableなどの文字どおりの意味に限定すること。
\item [He is a man who(彼は~というような男だ)]よくある種類の冗長な表現;ルール13を参照。
\begin{quote}
    He is a man who is very ambitious.
    
    (彼は非常に野心的な男だ)
    
    He is very ambitious.
    
    (彼は非常に野心的だ)
    
    Spain is a country which I have always wanted to visit.
    
    (スペインは私がずっと行きたいと思っている国だ)
    
    I have always wanted to visit Spain.
    
    (私はずっとスペインに行きたいと思っていた)
\end{quote}
\item [However(しかしながら)]nevertheless(それにもかかわらず)という意味の場合は、それが含まれる文や節の最初に来てはならない。
\begin{quote}
    The roads were almost impassable. However, we at last succeeded in
reaching camp.

(道路はほとんど通行できなかった。しかしながら、私たちはとうとうキャンプに到達できた)

The roads were almost impassable. At last, however, we succeeded
in reaching camp.

(道路はほとんど通行できなかった。それにもかかわらず、私たちはとうとうキャンプに到達できた)
\end{quote}
howeverが最初にくる場合、それはin whatever way(どんな方法であれ)またはto whatever extent(どの程度であれ)という意味だ。
\begin{quote}
    However you advise him, he will probably do as he thinks best.
    
    (あなたがどんなに彼に助言をしても、たぶん彼は自分が一番だと思うところを為すだろう)
    
    However discouraging the prospect, he never lost heart.
    
    (見通しがどんなに思わしくなかったとしても、彼は決して落胆しなかった)
\end{quote}
\item [Kind of(ある種の)]
(形容詞や動詞の前における)ratherの代わりに使ったり、くだけた体裁の場合を除いては(名詞の前における)something
likeの代わりに使ってはならない。文字どおりの意味に限ること:``Amber is a
kind of fossil resin;''(琥珀は化石樹脂の一種だ) ``I dislike that kind
of notoriety''(私はその手の悪評は好まない)。同じことがsort
ofについてもいえる。
\item [Less(分量が少ない)]これをfewer(数が少ない)と誤用してはならない。
\begin{quote}
    He had less men than in the previous campaign.
    
    (彼は前回の戦役よりも少ない兵士しか率いていなかった)
    
    He had fewer men than in the previous campaign.
    
    (彼は前回の戦役よりも少ない兵士しか率いていなかった)
\end{quote}
Lessは分量を指し、fewerは数を指す。``His troubles are less than
mine''は``His troubles are not so great as
mine.''(彼の問題は私の問題ほどひどくない)という意味だ。``His troubles
are fewer than mine''は``His troubles are not so numerous as
mine.''(彼の問題は私の問題ほど数が多くない)という意味だ。しかしながら、``The
signers of the petition were less than a
hundred,''(請願の署名は100に満たなかった)と言うのは、100のような丸められた数字は集合名詞のようなものであり、lessは分量や額が少ないという意味だと解釈されるので、正しい。
\item [Line, along these lines(線、この線で)]course of procedure, conduct,
thought(手続き、行動、考え方の指針)という意味でのlineは許容されるが、特にalong
these
lines(この線に沿って)というフレーズで多用されすぎているので、新鮮味やオリジナリティを求める書き手は完全に無視したほうがよい。
\begin{quote}
    Mr.~B. also spoke along the same lines.
    
    (B氏もまた同じ線で話をした)
    
    Mr.~B. also spoke, to the same effect.
    
    (B氏もまた同じ趣旨の話をした)
    
    He is studying along the line of French literature.
    
    (彼はフランス文学の概略に沿って研究をしている)
    
    He is studying French literature.
    
    (彼はフランス文学を研究している)
    \end{quote}
\item [Literal, literally (文字どおり)]誇張しようとして、または暴力的な暗喩をしようとして、しばしば誤って用いられる
\begin{quote}
    A literal flood of abuse(文字どおり、罵詈雑言の洪水)
    
    A flood of abuse(洪水のような罵詈雑言)
    
    Literally dead with fatigue(文字どおり疲労で死んだようになって)
    
    Almost dead with fatigue (dead tired)(疲労でほとんど死んだようになって(死ぬほど疲れて))
\end{quote}
\item [ Lose out(負ける) ]loseよりも語勢を強める意図で使われるが、実際にはそのありきたりさゆえに、より弱くなる。try
out, win out, sign up, register
upも同様。outおよびupは、さまざまな動詞と結合して慣用表現を形作る:find
out, run out, turn out, cheer up, dry up, make
upその他で、それぞれがもとの動詞から意味の違いを読み取れる。Lose
outはそうではない。
\item [Most(ほとんどの)] almost(ほぼ)という意味で使ってはならない。
\begin{quote}
    Most everybody(ほとんどの誰もが)
    
    Almost everybody(ほとんど誰もが)
    
    Most all the time(ほとんどのずっと)
    
    Almost all the time(ほとんどずっと)
\end{quote}
\item [Nature(性質)]character(性格)のように使われ、多くの場合は単に冗長だ。
\begin{quote}
    Acts of a hostile nature(敵対的な性質の行為)
    
    Hostile acts(敵対的な行為)
\end{quote}
``a lover of nature;''(自然を愛する者) ``poems about
nature''(自然についての詩)のような表現でしばしばあいまいに使われる。より具体的な記述がそれに続くのでない限り、その詩が自然の景観に関するものなのか、田園生活に関するものなのか、日没に関するものなのか、人跡未踏の荒野に関するものなのか、リスの習性に関するものなのか、読み手には分からない。
\item [Near by(近くの)]副詞的なフレーズで、いまだに良い英語だとは完全に認められていないが、close
byおよびhard
byのアナロジーによって正当化されているようだ。Nearまたはnear at
handも、少なくとも同程度には良い。
\par 形容詞として使ってはならない;neighboringを使うこと。
\item [Oftentimes, ofttimes(しばしば)] 古風な形式で、もうあまり使われていない。現代的な言葉はoften。
\item [One hundred and one(多数の)]これとこれに似た表現では、オールドイングリッシュ(Old English:
古期英語)の時代からの英語散文の一定不変の用法に従って、andを省かずにおく。
\item [One of the most(最も~なもののひとつ)]この定型でエッセイやパラグラフを始めるのは避けること。例えば``One
of the most interesting developments of modern science is,
etc.;''(現代科学の最も興味深い発展のひとつに、云々)や``Switzerland is
one of the most interesting countries of
Europe''(スイスはヨーロッパの最も興味深い国のひとつである)のような表現は避ける。これは決して間違ってはいない;単に陳腐で、一見強そうだが実は弱いというだけだ。
\item [People(人々)]The people(人々)は政治的な用語であり、the
public.(公衆)と混同してはならない。peopleが生み出すのは、政治的な協賛や抵抗だ;publicが生み出すのは、芸術的な賞賛や商業的な後援だ。\par eopleという語は、personsの代わりに数詞と共に使ってはならない。もし``six
people''のうち5人がいなくなったなら、何人の``people''が残るだろうか?
\item [Phase(段階)] 移り変わりや発展の段階を意味する:``the phases of the
moon;''(月相) ``the last
phase''(最終段階)。aspect(面)やtopic(話題)の意味で使ってはならない。
\begin{quote}
 Another phase of the subject(主題のもうひとつの面)
 
 Another point (another question)(もうひとつのポイント(もうひとつの疑問))
\end{quote}
\item [Possess(所有する、占有する)]haveやownの単なる代替として使ってはならない。
\begin{quote}
    He possessed great courage.
    
    (彼は大変な勇気を所有していた)
    
    He had great courage (was very brave).
    
    (彼は大変な勇気を持っていた(とても勇敢だった))
    
    He was the fortunate possessor of
    (彼は幸運にも~の所有者だった)
    
    He owned
    (彼は~を所有していた)
\end{quote}
\item [Respective, respectively(それぞれ)] これらの語は通常、省略してよく、そうしたほうがよい。
\begin{quote}
    Works of fiction are listed under the names of their respective
authors.

(フィクションの作品はそれぞれの作者の名前に基づいて列挙されている)

Works of fiction are listed under the names of their authors.

(フィクションの作品は作者の名前に基づいて列挙されている)

The one mile and two mile runs were won by Jones and Cummings
respectively.

(1マイル走と2マイル走はそれぞれJonesとCummingsが勝利した)

The one mile and two mile runs were won by Jones and by Cummings.

(1マイル走と2マイル走はJonesとCummingsが勝利した)
\end{quote}
幾何学的な証明のように、ある種の公式なライティングでは、respectivelyを使うことが必須であるかもしれないが、通常の主題についてのライティングで使われるべきではない。
\item[So(とても)]ライティングでは、soを強調のために使うのは避けること:``so
good''(とても良い);``so warm''(とても暖かい);``so
delightful''(とても喜ばしい)。
\par 節を導入するためにsoを使うことについては、ルール4を参照。
\item[Sort of(ある種の)]Kind ofを参照。
\item[State(述べる)]単にsay, remarkの代わりとして使ってはならない。``He refused to
state his
objections.''(彼は反対意見を述べることを拒否した)のように、余すところなく完全に、または明らかに表現する(express
fully or clearly)という意味に限定すること。
\item[Student body(全学生)]students(学生)という単純な語以上の意味を持たない、不必要で不格好な表現だ。
\begin{quote}
    A member of the student body (全学生のメンバー)
    
    A student(学生)
    
    Popular with the student body(全学生の間で人気の)
    
    Liked by the students(学生に人気の)
    
    The student body passed resolutions.(全学生は決議案を通過させた)
    
    The students passed resolutions.(学生は決議案を通過させた)
\end{quote}
\item [System(システム)]必要がないのによく使われる。
\begin{quote}
    Dayton has adopted the commission system of government.
    
    (Daytonは政府の委任システムを採用している)
    
    Dayton has adopted government by commission.
    
    (Daytonは委任の点では政府を受け入れている)
    
    The dormitory system(寮制)
    
    Dormitories(寮)
\end{quote}
\item[Thanking you in advance(先に感謝の言葉を述べておく)]これは書き手がこんなことを意図しているように読める:``It will not
be worth my while to write to you
again.''(私が時間を費してもう一度あなたに書くだけの価値はないだろう)。単にこう書けばよい:``Thanking
you,''(感謝する)。そしてあなたが求めた厚意が得られたなら、感謝状を送ればよい。
\item[They(彼ら)]よくある間違いは、先行詞がeach, each one, everybody, every one,
many a
manのような配分詞表現であるときに、複数形の代名詞を使うことだ。これらは一人の人間以上の意味があるものの、代名詞は単数でなければならない。これと同様に、正当性はさらに低くなるが、anybody,
any one, somebody, some oneという先行詞について、不格好な``he or
she''を避けるためか、またはどちらにも偏らないという意図を持って、複数形の代名詞を使うことが挙げられる。恥ずかしがりやの話し手は、``A
friend of mine told me that they,
etc.''(私の友人が言うことには、彼らは云々)と言うことさえある。 \par 上記の語すべてについては、先行詞が女性形であるか、女性形でなければならない場合を除き、heを使う
\item[Very(とても)]この単語を使うのは控えめにしたほうがよい。強調が必要なときは、それ自体が力強い言葉を使うこと。
\item [Viewpoint(視点)]] view(視点)と書くこと。ただし使い方を誤ってはならない。多くの人が間違えているが、view(見解)やopinion(意見)の意味で使うのは間違いだ。
\item[While(一方)]この語をand,
butおよびalthoughという意味で見境なく使うのは避けること。接続詞の種類を増やしたいという単純な気持ちから、そしてどちらの接続詞がより適切か判断がつかないために、多くの書き手がこれをandまたはbutの代わりによく使う。このような使われ方をしている場合、一番いいのはセミコロンで置き換えることだ。
\begin{quote}
    he office and salesrooms are on the ground floor, while the rest
of the building is devoted to manufacturing.

(事務所と販売用の部屋は地階にあり、その一方で建物の他の部分は生産のための空間になっている)

The office and salesrooms are on the ground floor; the rest of the
building is devoted to manufacturing.

(事務所と販売用の部屋は地階にある;建物の他の部分は生産のための空間になっている)
\end{quote}
これを事実上althoughの同等品として使うことは、そのために文があいまいになったり不合理に陥ったりしなければ許容される。
\begin{quote}
    While I admire his energy, I wish it were employed in a better
cause.

(私は彼の元気には感心はするものの、それがもっとましな目的のために生かされることを願う)

\end{quote}
言い換えで示されているとおり、これは全体としては正しい:
\begin{quote}
I admire his energy; at the same time I wish it were employed in a
better cause.

(彼の元気には感心する;同時に、それがもっとましな目的のために生かされることを願う)
\end{quote}
比較してみてほしい:
\begin{quote}
    While the temperature reaches 90 or 95 degrees in the daytime, the
nights are often chilly.

(日中は温度が90ないし95度に達する一方で、夜中は肌寒いことが多い)

Although the temperature reaches 90 or 95 degrees in the daytime,
the nights are often chilly.

(日中は温度が90ないし95度に達するけれども、夜中は肌寒いことが多い)
\end{quote}
次のパラフレーズは、
\begin{quote}
    The temperature reaches 90 or 95 degrees in the daytime; at the
same time the nights are often chilly,

(日中は温度が90ないし95度に達する;それと同時に、夜中は肌寒いことが多い)
\end{quote}
なぜwhileを使うのが誤っているかを示している。
\par 一般的にいって、whileをduring the time
that(~という時期の間)という厳密に字義どおりの意味に限って使っても、書き手はうまく書ける。
\item[Whom(誰に)]he
saidやそれに似た表現の前にあるwhoの代わりに誤って用いられることがよくある。実際には続く動詞の主語だというのに。
\begin{quote}
    His brother, whom he said would send him the money
    
    (彼が言うには彼にお金を送る先(?)であるところの、彼の兄)
    
    His brother, who he said would send him the money
    
    (彼が言うには彼にお金を送ってくれるであろうところの、彼の兄)
    
    The man whom he thought was his friend
    
    (彼が考えた男は彼の友人だった)
    
    The man who (that) he thought was his friend (whom he thought his
friend)

(彼が考えた男は、彼が自分の友人であると考えた人物だった)
\end{quote}
\item[Worth while(有意義な)]あいまいな賛成と(notをつけて)不賛成の言葉として、あまりに使われすぎている。行為に対してのみ適用可能:``Is
it worth while to telegraph?''(電報を打つ価値があるだろうか?)
\begin{quote}
    His books are not worth while.
    
    (彼の本は価値がない)
    
    His books are not worth reading (not worth one's while to read; do
not repay reading).

(彼の本は読む価値がない(読む時間を費すに値しない;読むことが報いられない))
\end{quote}
worth whileを名詞の前で使うこと(``a worth while
story'')は、弁解の余地なく間違っている。
\item[Would(だったろう)]
第一人称の条件節が必要とするのはshouldであって、wouldではない。
\begin{quote}
    I should not have succeeded without his help.
    
    (彼の助けなしには私は成功しなかっただろう)
\end{quote}
過去形の動詞の後の間接引用におけるshallの相当語句は、shouldであって、wouldではない。
\begin{quote}
    He predicted that before long we should have a great surprise.
    
    (近いうちに我々は思いがけない大きな出来事を迎えるだろうと彼は予言した)
\end{quote}
習慣的または何度も繰り返す行為を表現するためには、wouldなしの過去形で通常は十分であり、その簡潔さからいってよりはっきりしている。
\begin{quote}
    Once a year he would visit the old mansion.
    
    (1年に1度、彼はその古い屋敷を訪れていた)
    
    Once a year he visited the old mansion.
    
    (1年に1度、彼はその古い屋敷を訪れた)
\end{quote}
\end{description}

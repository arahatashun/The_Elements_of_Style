\chapter{形式に関するいくつかの問題}
\begin{description}
 \item [Headings(見出し)]原稿のタイトルや見出しの後には、空行かそれに相当する空白を置く。続くページでは、罫の入った紙を使っている場合は、最初の行から始める。
 \item [Numerals(数詞)]日付やその他の通し番号は、普通の単語として綴らない。適切な表記になるよう、数字かローマ記数法で記述する。
 \begin{quote}
     August 9, 1918 (1918年8月9日)
     Chapter XII(第XII章)
     Rule 3(ルール3)
     352d Infantry(第352歩兵隊)
 \end{quote}
 \item [Parentheses(括弧)]括弧に入った表現を含む文のうち、括弧記号の外側は、括弧内の表現がまるで存在しないかのように句読点を打つ。内側の表現は、それだけで独立しているかのようにみなして句読点を打つ。ただし内側の最後の句読点は、疑問符か感嘆符でない限り省略する。
 \begin{quote}
     I went to his house yesterday (my third attempt to see him), but
he had left town.

(私は昨日彼の家に行ったが(彼に会いに行くのはこれが3回目だ)、彼は町を離れた後だった)

He declares (and why should we doubt his good faith?) that he is
now certain of success.

(彼が宣言したことには(そして我々が彼の善き信仰を疑うことなどあろうか?)彼は今や自分自身の成功を確信している
 \end{quote}
 (完全に分離した表現や文が括弧で囲まれている場合、最後の句読点は最後の括弧記号の前にくる。)
\item [Quotations(引用)]
文書上の証拠として引用される正式な引用文は、コロンで導かれ引用符で囲まれる。
\begin{quote}
    The provision of the Constitution is: ``No tax or duty shall be
laid on articles exported from any state.''

(憲法にはこう定められている:「どの州から輸出される物品にも、租税や関税を課してはならない。」
\end{quote}
文法上同格である引用文や、動詞の直接目的語である引用文は、直前にカンマが来て引用符で囲まれる。
\begin{quote}
    I recall the maxim of La Rochefoucauld, ``Gratitude is a lively
sense of benefits to come.''

(私はLa
Rochefoucauldの、「感謝の念は、来たる利益の確かな感触である」という格言を思い出す。)

Aristotle says, ``Art is an imitation of nature.''

(アリストテレスは「芸術は自然の模倣である」と言った。)
\end{quote}
詩句の1行丸ごとかそれ以上の引用は、新しい行でセンタリングして始めるが、引用符で囲むことはしない。
\begin{quote}
    Wordsworth's enthusiasm for the Revolution was at first
unbounded:

(ワーズワースの革命に対する熱狂は当初限りなかった:)

\centering
Bliss was it in that dawn to be alive,

(生きて迎える曙光の至福よ、)

But to be young was very
heaven!

(長く生きることはこの上ない幸せ!)
\end{quote}
thatによって導入された引用は、間接話法の中にある状態と同様にみなされ、引用符で囲まれない。
\begin{quote}
    Keats declares that beauty is truth, truth beauty.
    
    (美は真なり、真は美とキーツは言い切っている)
\end{quote}
諺などでよく知られた表現および文学に由来する身近なフレーズは、引用符を必要としない。
\begin{quote}
    These are the times that try men's souls.
    
    (今こそ我々の魂が試されるときなのだ(訳注:Thomas Paine, ``The
Crisis'', 1776-83))

He lives far from the madding crowd.

(群集からはるかに離れて彼は生きる(訳注:Thomas Hardy, ``Far From
the Madding Crowd'', 1874))
\end{quote}
口語と卑語・俗語についても同様のことがいえる。
\item [References(リファレンス、参考文献)]
正確なリファレンスが必要とされる学術的な著作物では、何度も現れるタイトルは省略形で示し、巻末に完全形のアルファベット順一覧を載せる。一般的な慣習として、リファレンスは括弧に入れるか脚注にするかして、本文中にそのまま書くことはしない。act,
scene, line, book, volume,
page(幕、場、行、冊、巻、頁)という語は、これらのうちただ1種類によって参照するとき以外は省略する。句読点は下記のように打つ。
\begin{quote}
    In the second scene of the third act
    
    (第三幕の第二場において)
    
    In III.ii (still better, simply insert III.ii in parenthesis at
the proper place in the sentence)


(III.iiにおいて(さらに良い書き方は、単に文中の適切な箇所に、括弧に包んでIII.iiと挿入する))

After the killing of Polonius, Hamlet is placed under guard (IV.
ii. 14).

(ポロニウスを殺害した後、ハムレットは警備兵の監視下におかれる(IV.
ii. 14))

2 Samuel i:17-27

Othello II.iii 264-267, III.iii. 155-161
\end{quote}
\item [Titles (タイトル、書名)]
文学的作品のタイトルは、学術的な用法では、イニシャルを大文字にしたイタリックにすることが望ましい。編集者や出版社によって用法は異なり、イニシャルを大文字にしたイタリックを使うところもあれば、イニシャルを大文字にしたローマンに引用符を付けたり付けなかったりするところもある。執筆している雑誌や媒体が異なる慣習に従う場合を除き、イタリックを使うこと(原稿上では該当箇所に下線を引くことで示される)。タイトルの前に所有格を置く場合は、タイトルの最初のAまたはTheは省略する。
\begin{quote}
    The Iliad; the Odyssey; As You Like It; To a Skylark; The
Newcomes; A Tale of Two Cities; Dicken's Tale of Two Cities.

(イリアッド;オデュッセイア;お気に召すまま;雲雀に;ニューカム家の人々;二都物語;ディケンズの二都物語)
\end{quote}
\end{description}